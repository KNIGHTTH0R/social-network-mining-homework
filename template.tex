% chose the documnet class as article
\documentclass[11pt]{article}

% basic setteings
\topmargin=-0.45in
\evensidemargin=0in
\oddsidemargin=0in
\textwidth=6.5in
\textheight=9.0in
\headsep=0.25in

% create a new command 
\newcommand{\hwtitle}{Homework\ \#1}
\newcommand{\student}{Zeng Xiangrong}
\newcommand{\teacher}{Professor Hao Wang}

% this package is used for setting header and footer
\usepackage{fancyhdr}
\pagestyle{fancy}

\begin{document}
% set the title page
\begin{titlepage}
\begin{center}
    \textbf{}
    \\[4.5cm]
    \textbf{\huge \hwtitle}
    \\[0.3cm]
    \textnormal{Due on}
    \today
    \\[0.4cm]
    \emph{\teacher}
    \\[8cm]
    \textbf{\Large \student}
\end{center}
\end{titlepage}

% setting the header and footer
\fancyhf{} % clear the header and footer
\lhead{\student}
\chead{\hwtitle}
\rhead{problem \thesection}
\renewcommand{\footrulewidth}{0.4pt}
\cfoot{\thepage}

% setting the section style
\setcounter{section}{0} % set the counter start from 0
% set the section style to "problem 1" and increase correctly
\renewcommand{\section}{
    \stepcounter{section}
    \begin{flushleft}
        \raggedright \Large\textbf{Problem \thesection\\}
    \end{flushleft}}

% main contex
\section
% put the main text of every problem here
Proof: Lemma2. In any directed graph, the summation of in-degrees is equal to the summation of out-degree.
\subsection*{Solution}
\clearpage % end current page
\section
% put the main text of every problem here
Proof: Lemma2. In any directed graph, the summation of in-degrees is equal to the summation of out-degree.
\subsection*{Solution}
My answer is ...

\end{document}
